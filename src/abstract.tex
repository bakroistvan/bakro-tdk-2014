%--------------------------------------------------------------------------------------
% Feladatkiiras (a tanszeken atveheto, kinyomtatott valtozat)
%--------------------------------------------------------------------------------------
\clearpage
\begin{center}
\large
\textbf{Kivonat}
\end{center}
\addcontentsline{toc}{chapter}{Kivonat}

Atomerő mikroszkóppal való fémezett felületű minta felületi töltéssûrûségének mérése során kritikus a tû és a minta közötti kapacitás ismerete.
A kapacitás értékét közelítések mellett lehetséges analitikusan kifejezni.
Numerikus szimulációval pontosabban ismerhetjük az értékét, ezáltal a felbontás nõ és a mérési zaj csökken.

A szimuláció során a lehetséges párhuzamosításokat felhasználva a számítási idõt elfogadhatóra csökkentettük,
ami akár online feldolgozást is lehetõvé teszi.

Atomerő mikroszkóppal való fémezett felületű minta felületi töltéssűrűségének
mérése során kritikus a tű és a minta közötti kapacitás ismerete.
A kapacitás értékét közelítések mellett lehetséges analtikusan kifejezni.
Numerikus szimulációval pontosabban ismerhetjük az értékét, ezáltal nagyobb
felbontás is érhető el.
A szimuláció során a lehetséges párhuzamosításokat felhasználva a számítási időt
elfogadhatóra csökkentettük.


\clearpage
\begin{center}
\large
\textbf{Abstract}
\end{center}
\addcontentsline{toc}{chapter}{Abstract}

The measurement of the surface charge density can be achived by Atomic Force Microscope (AFM).
The common method is the two-pass technic, which main goal is to seperate the net force acting upon the tip into components.
To do so we have to express the tip-sample capacitance, which is critical respectivly to the measurements accuracy.

The value of the capacitance can be expressed analitically, but with some strong neglecting.

We are presenting the accelerated simulation of this capacitance using GPU's programmed in
OpenCL's parallel framework.
